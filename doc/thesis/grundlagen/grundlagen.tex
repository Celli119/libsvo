\chapter{Grundlagen}


\section{Volumendaten}
Volumendaten k�nnen durch dreidimensionale, �quidistante Gitter beschrieben werden. Die Kreuzungspunkte des Gitters werden \textit Voxel (Volumen-Pixel) genannt. Jeder Voxel kann einen einzelnen skalaren Wert, wie beispielsweise Dichte oder Druck, oder mehrere skalare Werte wie Farbe in Kombination mit Richtungsinformationen enthalten. Dadurch eignet sich diese Darstellung zur Repr�sentation eines �quidistant gesampelten Raumes, der nicht homogen gef�llt ist. Durch die uniforme Unterteilung des Raumes ist die Position und die Ausdehnung eines jeden Voxels implizit in der Datenstruktur enthalten und muss daher nicht gespeichert werden.\\
Volumendaten werden vorwiegend in der Medizin, beispielsweise als Ausgabe der Magnetresonanztomographie oder in der Geologie zum Abbilden der Ergebnisse von Reflexionsseismikverfahren verwendet.\\
Um eine hinreichende Aufl�sung der Volumenrepr�sentation zu gew�hrleisten sind gro�e Datenmengen erforderlich. Ein mit $512^3$ Voxeln aufgel�stes Volumen, dessen Voxel jeweils einen mit 4 Byte abgebildeten Skalar enthalten, belegt bereits 512 Megabyte. Verdoppelt man die Aufl�sung auf $1024^3$ verachtfacht sich der Speicherbedarf auf 4 Gigabyte. Allerdings enthalten Volumendaten in der Regel einen gro�en Anteil an homogenen Bereichen, die jedoch durch ein regul�res Gitter als viele Einzelwerte abgebildet werden m�ssen. Daher gibt es Datenstrukturen die ausgehend von dem regul�ren Gitter eine hierarchische Struktur erzeugen um diese Bereiche zusammenzufassen.


\section{Octrees}
Ein Octree ist eine raumteilende, rekursive Datenstruktur. Ein initiales, kubisches Volumen wird in acht gleich gro�e Untervolumen geteilt. Die Teilung wird f�r jedes Untervolumen fortgef�hrt, f�r das ein h�herer Detailgrad ben�tigt wird. Bereiche die homogene Daten enthalten oder leer sind k�nnen von der Unterteilung ausgeschlossen werden, wodurch eine wesentlich kompaktere Darstellung der Daten gegen�ber konventionellen Volumendaten erreicht werden kann. F�r jedes Volumen/Voxel existiert ein Verweis auf die ihn unterteilenden Untervolumen wodurch eine Baumstruktur entsteht. In der Regel besitzt jeder Voxel acht Kinder (\textit{innerer Knoten}) oder kein Kind \textit{Blatt-Knoten}. Mit jeder Unterteilung verdoppelt sich die Aufl�sung der abbildbaren Information auf jeder Achse. Die Gr��e eines Voxels kann mit $ 2^{-d} $ bestimmt werden wobei $d$ die Tiefe des Voxels in der Baumstruktur, beginnend mit $d=0$ f�r die Wurzel, ist. 
Jeder Voxel kann ein oder mehrere Attribute, wie Farbe, Richtungsvektor oder ... speichern. Die Attribute eines �bergeordneten Voxels ergeben sich dabei im einfachsten Fall aus dem Mittel der Attribute seiner untergeordneten Voxel, vergleichbar mit der Erzeugung von \textit{Mipmaps}. Somit enth�lt jeder Voxel einen seiner Gr��e entsprechend Detailgrad an Attributinformation. Der wesentliche Vorteil von Octrees gegen�ber texturierten Dreiecksnetzen ist somit, das die Datenstruktur das LOD-Problem nicht nur f�r Attribute (Texturen), sondern auch f�r Geometrie auf elegante Weise l�st - alles in einem einfachen Algorithmus. Der Octree ist also beides: Geometrie und Textur.\\
%http://www.tomshardware.com/reviews/voxel-ray-casting,2423-5.html

\section{Sparse Octrees}
F�r einige Anwendungen sind nur bestimmte Auspr�gungen der in den Voxeln gespeicherten Werte von Interesse. Beispielsweise werden beim Iso-Surface-Rendering nur Voxel mit einem bestimmten Dichtewert als opake Oberfl�che dargestellt. Ist dies der Fall kann die Datenstruktur weiter ausged�nnt werden, in dem nur noch Voxel gespeichert werden die zur Oberfl�che beitragen. Somit k�nnen innere Knoten weniger als acht Kinder haben. Eine solche ~Volumenrepr�sentation eignet sich ebenso zur Darstellung anderer opaker Oberfl�chen wie diskretisierte Dreiecksnetze, Punktwolken oder H�henfelder und wird als Sparse Octree oder Sparse Voxel Octree bezeichnet.\\ 

\section{Raycasting}
Durch die Eigenschaften eines voll besetzten Octrees ist seine Darstellung im Speicher implizit vorgegeben. Da jeder Elternknoten genau acht Kinder besitzt kann durch seine Position in der serialisierten Struktur implizit auf seine Kindknoten geschlossen werden. !!! Durch die Regel $ C(P,n) = 8*P+n $ berechnet werden (!!!BILD). Dabei ist $P$ die Position des Elternknotens, $n$ die Nummer des Kindes und $C$ die Position des Kindknotens.!!!\\
Durch die variierende Anzahl der Kinder in einer Sparse Octree Struktur ist keine solche Regel vorhanden. Vielmehr muss jeder Knoten einen Verweis auf die vorhandenen Kindknoten speichern.











