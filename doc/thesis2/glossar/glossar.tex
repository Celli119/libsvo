
\chapter{Glossar}


\textbf{Aliasing}\\
Beim Abtasten von Daten mit zu geringer Abtastfrequenz entstehende Fehler. Diese f�hren in Dar\-stel\-lungen zu Mustern und Bildartefakten, die nicht in den Originaldaten enthalten sind. 
\\
\\
\textbf{Ambient Occlusion}\\
Eine Shading-Methode zur Simulation globaler Beleuchtungsdaten. Dabei wird ein Ma� der Verdeckung eines Teils der Daten durch benachbarte Bereiche ermittelt und auf die Beleuchtungdaten angewendet. 
\\
\\
\textbf{Antialiasing}\\
Techniken zur Minimierung von Aliasing.
\\
\\
\textbf{Bounding Box}\\
Ein Quader oder W�rfel, der zur Optimierung von Berechnungen komplexere, in diesem Volumen ent\-haltene Daten repr�sentiert.
\\
\\
\textbf{Culling}\\
Ein Verfahren, bei dem mit Hilfe von Sichtbarkeits- und Verdeckungsinformationen eine Reduzierung der zur Bilderzeugung zu verarbeitenden Daten erreicht werden kann. 
\\
\\
\textbf{GPGPU} (General-purpose computing on graphics processing units)\\
Bezeichnet die Verwendung eines Grafikprozessors zur L�sung auch andere algorithmische Probleme.
\\
\\
\textbf{Mipmaps}\\
Eine LOD-Technik f�r Texturen, die zur Vermeidung von Aliasing-Artefakten verwendet wird. Eine Mipmap besteht aus einer hierarchischen Anordnung von unterschiedlich aufgel�sten Repr�sentationen der Ausgangsdaten. W�hrend der Bilderzeugung werden diejenigen Repr�sentationen ausgew�hlt, die der Bildaufl�sung am besten entsprechen.  
\\
\\
\textbf{Popping-Artefakt}\\
Eine pl�tzliche und deutlich sichtbare �nderung der Darstellung. Oft entstehen diese Artefakte aufgrund eines �bergangslosen Wechsels der, f�r die Darstellung verwendeten Detailstufen. 
\\
\\
\textbf{Raycasting}\\
Eine Methode zur Visualisierung von Daten. Dabei wird pro darzustellendem Bildpunkt der Weg von mindestens einem Strahl durch die r�umlichen Daten berechnet. Dabei werden beispielsweise Farb- und Beleuchtungswerte ermittelt.  
\\
\\
\textbf{Speicherkoh�renz}\\
Eine Eigenschaft von Speicherzugriffen, die sich aus der Entfernung zwischen aufeinanderfolgenden Zugriffen ergibt. F�r Zugriffe auf benachbare Speicherregionen, ist die Speicherkoh�renz hoch. Operationen mit hoher Speicherkoh�renz lassen sich oft effizienter ausf�hren, als solche mit niedriger Speicherkoh�renz.
\\
\\
\textbf{UV-Koordinaten}\\
Eine Fl�chenkoordinate, die zum Abbilden von Texturdaten auf 3D-Objekte verwendet wird. 
\\
\\
\textbf{Client-Server-Architektur}\\
Beschreibt ein aus zwei Subsystemen bestehendes System. Der Server stellt dabei Dienste zur Verf�gung, die vom Client angefordert werden.