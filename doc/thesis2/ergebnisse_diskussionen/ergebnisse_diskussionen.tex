\chapter{Ergebnisse und Diskussion}


\section{�berblick}
\section{Verwendete Testszenen}
 

Zum Testen des Out-of-Core Systems wurden SVO-Strukturen f�r drei unterschiedliche Geometrien generiert (siehe Tabelle ...). 
Aus jeder Geometrie wurden zwei verschiedene SVO-Strukturen erstellt. Beide haben jeweils die selbe minimale Tiefe, aber eine unterschiedliche Treelet Gr��e. Die Tiefe betrug jeweils 12 bei Treelet Gr��en von 1kb und 4kb. Durch die Verkleinerung der Treelets bei gleicher minimaler SVO-Tiefe entstehen mehr Treelets und damit eine gr��erer Aufwand bei der Verwaltung. Die beiden Varianten sollen im folgenden verglichen werden, um die Skalierbarkeit des Out-of-Core Systems nach zu vollziehen. In Tabelle ... wird gezeigt... .




\begin{table}
\centering

\begin{tabular}{ l  r  r  r  r }
  \hline
  \textbf{Name} & \textbf{Anzahl Dreiecke} & \textbf{Treelet Gr��e} & \textbf{Anzahl Treelets} & \textbf{Anzahl Knoten} \\
  \hline
  David face & 1000 & 1kb & 12 & 1231561 \\
  David face & 1000 & 4kb & 12 & 1231561 \\
  \hline
  xyz Dragon & 1000 & 1kb & 12 & 1231561 \\
  xyz Dragon & 1000 & 4kb & 12 & 1231561 \\
  \hline  
  Lucy       & 1000 & 1kb & 12 & 1231561 \\
  Lucy       & 1000 & 4kb & 12 & 1231561 \\
  \hline
\end{tabular}

\caption{Math 500 Grades}\label{math500grades}
\end{table}
 
 
 


\section{Laufzeitanalyse}
Zum Testen wurden die Zeiten f�r die einzelnen Verarbeitungsschritte des Out-of-Core Systems gemessen. W�hrend der Messung wurde die Bildsynthese deaktiviert, um ausschlie�lich den Einfluss der Anzahl der Treelets auf den Ver\-walt\-ungsaufwandes zu untersuchen. Der SVO wurde �ber einen Zeitraum von 60 Sekunden aus verschiedenen Perspektiven analysiert und verfeinert. Durch die kontinuierliche Ver�nderung der Perspektive muss das System permanent neue Teile des Octrees anfordern, w�hrend es andere verwerfen muss. Damit bei der Verfeinerung auch Treelets in hohen Octree Tiefen angefordert werden, wird zum testen eine Kamerafahrt verwendet die ein Ma� an Koh�renz zwischen den Perspektiven zu gew�hrleisten.



\section{Einschr�nkungen und Verbesserungen}

\subsection{Serverseitige Aktualisierung}
Die in Abschnitt \ref{sec:serverseitige_aktualisierung} beschriebene Zusammenfassung der zu kopierenden Incore-Buffer-Slots wirkt sich deutlich auf die zum �bertragen der ge�nderten Speicherbereiche ben�tigte Zeit aus. Wie beschrieben arbeitet der Ansatz am besten wenn die zu transferierenden Speicherbereiche 
Mit steigender Fragmentierung des Incore-Buffers sinkt die Einsparung jedoch und schwankt stark. Bei einem vorgegebenen Verh�ltis zu aktualisierenden Slots von 20\% etwa schwankt die Einsparung zwischen 10\% bis 92\% und lag �ber einen Zeitraum von 60 Sekunden im Mittel bei etwa 43\%. Die hier examplarisch genannten Werte sind jedoch wenig aussagekr�ftig, da das Verhalten des Ansatzes nicht nur von der Anzahl der Slots, der Gr��e und dem Fragmentierungsgrad des Incore-Buffers abh�ngt, sondern auch von der Geometrie und der Kameraposition �ber die Zeit. Eine Verbesserung zum einzelnen Kopieren der Slots ist jedoch erkennbar.

\subsection{Verwendung von OpenGL Texturen als Buffer}
...