\chapter{Grundlagen}


\section{Volumendaten}
Volumendaten k�nnen durch dreidimensionale, �quidistante Gitter beschrieben werden. Die Kreuzungspunkte des Gitters werden \textit Voxel (Volumen-Pixel) genannt. Jeder Voxel kann einen einzelnen skalaren Wert, wie beispielsweise Dichte oder Druck, oder mehrere skalare Werte wie Farbe in Kombination mit Richtungsinformationen enthalten. Dadurch eignet sich diese Darstellung zur Repr�sentation eines �quidistant gesampelten Raumes, der nicht homogen gef�llt ist. Durch die uniforme Unterteilung des Raumes ist die Position und die Ausdehnung eines jeden Voxels implizit in der Datenstruktur enthalten und muss daher nicht gespeichert werden.\\
Volumendaten werden vorwiegend in der Medizin, beispielsweise als Ausgabe der Magnetresonanztomographie oder in der Geologie zum Abbilden der Ergebnisse von Reflexionsseismikverfahren verwendet.\\
Um eine hinreichende Aufl�sung der Volumenrepr�sentation zu gew�hrleisten sind gro�e Datenmengen erforderlich. Ein mit $512^3$ Voxeln aufgel�stes Volumen, dessen Voxel jeweils einen mit 4 Byte abgebildeten Skalar enthalten, belegt bereits 512 Megabyte. Verdoppelt man die Aufl�sung auf $1024^3$ verachtfacht sich der Speicherbedarf auf 4 Gigabyte. Allerdings enthalten Volumendaten in der Regel einen gro�en Anteil an homogenen Bereichen, die jedoch durch ein regul�res Gitter als viele Einzelwerte abgebildet werden m�ssen. Daher gibt es Datenstrukturen die ausgehend von dem regul�ren Gitter eine hierarchische Struktur erzeugen um diese Bereiche zusammenzufassen.


\section{Octrees}
Ein Octree ist eine raumteilende, rekursive Datenstruktur. Ein initiales, kubisches Volumen wird in acht gleich gro�e Untervolumen geteilt. Die Teilung wird f�r jedes Untervolumen fortgef�hrt, bis eine maximale Tiefe beziehungs\-weise ein maximaler Unterteilunggrad erreicht ist. Mit jeder Tiefen\-stufe des Octrees verdoppelt sich die Aufl�sung der abbildbaren Information auf jeder Achse. Die Gr��e eines Voxels kann mit $ 2^{-d} $ bestimmt werden wobei $d$ die Tiefe des Voxels in der Baumstruktur, beginnend mit $d=0$ f�r die Wurzel, ist. F�r vollbesetzte Octrees ist eine Darstellung im Speicher implizit vorgegeben. Da jeder Elternknoten genau acht Kinder besitzt kann durch seine Position in einer angenommenen, seriellen Struktur implizit auf seine Kind\-knoten geschlossen werden. Dabei kann die Position jedes Kindes eines Knotens durch $ C(P,n) = 8*P+n $ berechnet werden (!!!BILD) wobei $P$ die Position des Elternknotens, $n$ die Nummer des Kindes (beginnend mit 1) und das resultierende $C$ die Position des Kindknotens ist.\\
Bereiche, die homogene Daten enthalten oder leer sind, k�nnen jedoch von der Unterteilung ausgeschlossen werden, wodurch eine wesentlich kompaktere Darstellung der Daten gegen�ber konventionellen Volumendaten erreicht werden kann. F�r jedes Volumen/Voxel muss dann ein Verweis auf die ihn unterteilenden Untervolumen existieren. In der Regel besitzt jeder Voxel eines solchen Octrees acht Kinder (\textit{innerer Knoten}) oder kein Kind (\textit{Blatt-Knoten}). Die, im ung�nstigsten Fall zu speichernden sieben leeren Knoten, sind bei dieser Darstellung n�tig, um homogene Bereiche innerhalb des Eltern-Voxels zu kodieren.\\
Jeder Voxel kann ein oder mehrere Skalare speichern. Oft werden diese Werte nicht direkt im Octree abgelegt um bei die Traversierung der Struktur m�glichst wenig Speicher lesen zu m�ssen. Stattdessen werden die Attributinformation in einem zus�tzlichen Attribut-Buffer abgelegt, in dem zu jeder Voxel-Position im Octree ein Tuple mit Attributinformation an der selben Stelle im Attribut-Buffer vorgehalten wird. Die Attribute eines �bergeordneten Voxels ergeben sich dabei im einfachsten Fall aus dem Mittel der Attribute seiner untergeordneten Voxel, vergleichbar mit der Erzeugung von \textit{Mipmaps}. Somit enth�lt jeder Voxel einen seiner Gr��e entsprechend Detailgrad an Attributinformation. Der wesentliche Vorteil des Octrees gegen�ber texturierten Dreiecksnetzen ist somit, das die Datenstruktur das LOD-Problem nicht nur f�r Attribute (Texturen), sondern auch f�r Geometrie l�st. Der Octree ist also beides: Geometrie und Textur.


\section{Sparse Octrees}
F�r einige Anwendungen sind nur bestimmte Auspr�gungen der in den Voxeln gespeicherten Werte von Interesse. Beispielsweise werden beim Iso-Surface-Rendering nur Voxel mit einem bestimmten Dichtewert als opake Oberfl�che dargestellt. Ist dies der Fall kann die Datenstruktur weiter ausged�nnt werden, in dem nur noch Voxel gespeichert werden die zur Oberfl�che beitragen. Somit k�nnen innere Knoten weniger als acht Kinder haben. Eine solche ~Volumenrepr�sentation eignet sich ebenso zur Darstellung anderer opaker Oberfl�chen wie diskretisierte Dreiecksnetze, Punktwolken oder H�henfelder und wird als Sparse Octree oder Sparse Voxel Octree bezeichnet. Durch die variierende Anzahl von Kindknoten existiert keine implizite Regel zum berechnen deren Positionen. Vielmehr muss jeder Knoten speichern welche Kindknoten vorhanden sind und wo sich diese im Speicher befinden. Liegen die Kindknoten jedes Voxels jeweils hintereinander im Speicher muss nur ein Verweis pro Elternknoten vorgehalten werden.\\
Da in Sparse-Voxel-Octrees nur Oberfl�chen gespeichert werden steigt der Speicherbedarf pro weiterer Tiefenstufe nur durchschnittlich um das vierfache (vgl. ESVORG)


\section{Raycasting}











