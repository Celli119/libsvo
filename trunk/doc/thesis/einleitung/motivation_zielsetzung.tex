\chapter{Einleitung}
\section{Motivation}
Seit ihrer Vorstellung in den sp�ten 70er Jahren (!!! REFERENCE) ist die Bildsynthese durch Rasterisierung von parametrisierten Dreiecken der Quasi-Standard f�r Echtzeitcomputergrafik. Diese Entwicklung wurde nicht zuletzt durch die Einf�hrung von dezidierter Hardware und offenen Standards, wie OpenGL m�glich. Der Vorteil von Dreiecken als Geometrieprimitiv ist, dass sich mit ihnen sehr effizient planare Fl�chen darstellen lassen. Dabei hat die Gr��e der abgebildeten Fl�chen keinen Einfluss auf den Speicherbedarf der Repr�sentation. In modernen Anwendungen, wie Spielen oder bei der Darstellung von hochaufl�senden 3d-Scanns ist dieser Vorteil jedoch immer weniger relevant, da der �berwiegende Teil des ben�tigten Speichers durch Texturen belegt wird. Diese werden ben�tigt, um die Fl�chen mit Details zu versehen, wie Farbe, Richtung und andere zur Beleuchtung ben�tigten Attribute. Dabei ist die Parametrisierung mit Textur\-koordinaten von komplex geformten Dreiecksnetzen nicht trivial und muss deshalb meist h�ndisch bewerkstelligt werden.\\
Bei der Rasterisierung von detaillierten Dreiecksnetzen mit hochaufgel�sten Texturen kommt es schnell zu Aliasing\-artefakten. Um diese zu reduzieren, werden von Dreiecksnetz und Texturen niedriger aufgel�ste, statische Versionen erzeugt (\textit{Level-of-Detail, LOD}). Zwischen diesen wird bei der Darstellung je nach Betrachtungsabstand gewechselt, was zu st�renden \textit{Popping}-Artifakten f�hrt. Dabei kann nur im seltensten Fall ein ideales Verh�ltnis zwischen Geometrie- und gegebener Bildaufl�sung gew�hrleistet werden. Das dynamische Erstellen von LOD-Stufen aus einem hochaufl�senden Dreiecksnetz ist nur mit hohen Rechen- oder Speicheraufwand dynamisch zu bewerkstelligen. Au�erdem muss das LOD-Problem f�r Geometrie und Texturdaten w�hrend der Erstellung und der Darstellung separat gel�st werden. Ein Nachteil des Rasterisierungsansatzes ist das Fehlen von globalen Informationen w�hrend der Fragmentgenerierung. Jedes Primitiv wird f�r sich behandelt ohne das globale Informationen zur Optimierung (\textit{Culling}) oder Beleuchtung (\textit{Global Illumination}) zur Verf�gung stehen.\\ \\
Die Generalisierung der Renderpipelines und die Einf�hrung von GPGPU-Hochsprachen wie OpenCL machen es m�glich die Frage nach geeignetem Geometrieprimitiv und Bildsyntheseverfahren neu zu stellen. Sparse Voxel Octree als Datenstruktur in Kombination mit \textit{Raycasting} als Algorithmus zur Bildsynthese bieten viele positive Eigenschaften. So vereinen Sparse Voxel Octrees Geometrie und Texturdaten in einer einzigen hierarchischen Datenstruktur. Durch Raycasting auf dieser Struktur kann das LOD-Problem von Geometrie und Textur gemeinsam pro Bildpunkt gel�st werden. Gleichzeitig wirkt der Octree als Beschleunigungsstruktur, so dass w�hrend des Traversierens nur die Teile der Struktur durchlaufen werden, die zur Bildsynthese beitragen. Eine Parametrisierung ist nicht notwendig, da jedes Voxel seine eigenen Attributinformationen spei\-chert, die f�r seine Gr��e in optimaler Aufl�sung vorliegen.\\\\
Denoch gibt es einige Herausforderungen die bei der verwendung von Sparse Voxe Octrees gel�st werden m�ssen. Sparse Voxe Octrees ben�tigen viel Speicher. Die Menge an Arbeitsspeicher aktueller Grafikkarten gen�gt  um eine SVO-Struktur in hinreichender Aufl�sung zu speichern. Will man m�glichst viele Details oder sehr gro�e Strukturen abbilden k�nnen leicht mehrere Gigabyte Daten anfallen.\\
Da Sparse Voxel Octrees erst in j�ngster Zeit in den Fokus von Wissenschaft und Industrie ger�ckt sind, sind sie in Programmen zur Erstellung von 3D-Inhalten noch nicht angekommen. Daher lohnt es �ber die Generierung von Sparse-Voxel-Octrees-Strukturen aus anderen 3D-Datenformaten nachzudenken. 

...
\textbf{Probleme:} Trotz Sparse enorme Datenmenge
\textbf{Probleme:} keine Tools bzw. generalisierte Pipeline zur Erstellung von SVO-Content vorhanden\\


\section{Zielstellung}
In der vorliegenden Arbeit 



\textbf{Zeilstellung 2:} Entwicklung eines Out-Of-Core Ansatzes basierend auf Segmentierung der SVO Daten und adaptives refinement

\textbf{Zeilstellung 1:} Entwicklung eines Templates zur Generierung von SVO aus unterschiedlichen Datenvorlagen (Dreiecke, Pointclouds, Heightmaps, volumen).\\
\\
  
  