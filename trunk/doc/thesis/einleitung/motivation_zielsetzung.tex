\chapter{Einleitung}
\section{Motivation}
Seit ihrer Vorstellung in den sp�ten 70er Jahren (!!! REFERENCE) ist die Bildsynthese durch Rasterisierung von parametrisierten Dreiecken der Quasi-Standard f�r Echtzeitcomputergrafik. Diese Entwicklung wurde nicht zuletzt durch die Einf�hrung von dezidierter Hardware und offenen Standards, wie OpenGL m�glich. Der Vorteil von Dreiecken als Geometrieprimitiv ist, dass sich mit ihnen sehr effizient planare Fl�chen darstellen lassen, wobei die Gr��e der abgebildeten Fl�chen keinen Einfluss auf den Speicherbedarf der Repr�sentation hat. In modernen Anwendungen, wie Spielen oder bei der Darstellung von hochaufl�senden 3d-Scanns ist dieser Vorteil jedoch immer weniger relevant, da der �berwiegende Teil des ben�tigten Speichers durch Texturen belegt wird, welche die Fl�chen mit Details versehen. Dabei ist die Parametrisierung von komplex geformten Dreiecksnetzen nicht trivial und muss deshalb meist h�ndisch bewerkstelligt werden.\\
Bei der Rasterisierung von detaillierten Dreiecksnetzen mit hoch aufgel�sten Texturen kommt es schnell zu Aliasing\-artefakten. Um diese zu reduzieren, werden von Dreiecksnetz und Texturen niedriger aufgel�ste, statische Versionen erzeugt, zwischen denen bei der Darstellung je nach Betrachtungsabstand gewechselt wird, was zu st�renden \textit{Popping}-Artifakten f�hrt. Dabei kann nur im seltensten Fall ein ideales Verh�ltnis zwischen Geometrie- und gegebener Bildaufl�sung gew�hrleistet werden. Das Erstellen von \textit{Level-of-Detail}-Stufen (\textit{LOD}) aus einem hochaufl�senden Dreiecksnetz ist nur mit hohen Rechen- oder Speicheraufwand dynamisch zu bewerkstelligen. Au�erdem muss das LOD-Problem f�r Geometrie und Texturdaten w�hrend der Erstellung und der Darstellung separat gel�st werden. Ein Nachteil des Rasterisierungsansatzes ist das Fehlen von globalen Informationen w�hrend der Fragmentgenerierung. Jedes Primitiv wird f�r sich behandelt ohne das globale Informationen zur Optimierung (\textit{Culling}) oder Beleuchtung (\textit{Global Illumination}) zur Verf�gung stehen.\\ \\
Die Generalisierung der Renderpipelines und die Einf�hrung von GPGPU-Hochsprachen wie OpenCL machen es m�glich die Frage nach geeignetem Geometrieprimitiv und Bildsyntheseverfahren neu zu stellen. Sparse Voxel Octree als Datenstruktur in Kombination mit \textit{Raycasting} als Algorithmus zur Bildsynthese bieten viele positive Eigenschaften. So vereinen Sparse Voxel Octrees Geometrie und Texturdaten in einer einzigen hierarchischen Struktur. Durch Raycasting auf dieser Struktur kann das Problem der Wahl der Detailgrade von Geometrie und Textur gemeinsam pro Bildpunkt gel�st werden. Gleichzeitig wirkt der Octree als Beschleunigungsstruktur, so dass w�hrend des Traversierens nur die Teile der Struktur durchlaufen werden, die zur Bildsynthese beitragen. Eine Parametrisierung ist nicht notwendig, da jedes Voxel seine eigenen, f�r seine Gr��e optimal aufgel�sten, Attributinformationen speichert.\\
...
\textbf{Probleme:} keine Tools bzw. generalisierte Pipeline zur Erstellung von SVO-Content vorhanden\\
\textbf{Probleme:} Trotz Sparse enorme Datenmenge


\section{Zielstellung}

\textbf{Zeilstellung 1:} Entwicklung eines Templates zur Generierung von SVO aus unterschiedlichen Datenvorlagen (Dreiecke, Pointclouds, Heightmaps, volumen).\\
\\
\textbf{Zeilstellung 2:} Entwicklung eines Out-Of-Core Ansatzes basierend auf Segmentierung der SVO Daten und adaptives refinement
  
  