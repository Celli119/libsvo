\chapter{Zusammenfassung und Ausblick}


\section{Zusammenfassung}

In der vorliegende Arbeit wurde ein Out-of-Core Ansatz zur Darstellung von gro�en Sparse-Voxel-Octree-Strukturen entwickelt. Dieser arbeitet durch Segmentierung der Octree-Daten in Teilen fester Gr��e. Die Anpassung der darzustellenden Untermenge an Octree-Daten wurde durch ein Analysesystem realisiert das auf Abtastung des momentan vorgehaltenen Octrees basiert. Zur Erstellung von Inhalten wurde ein generalisiertes System zur Erstellung von Sparse Voxel Octrees aus unterschiedlichen Ausgangsformaten entworfen. Um die Ergebnisse der Erstellung und des Out-of-Core-Ansatzes zu validieren wurde der Raycasting-Algorithmus in OpenCL implementiert.\\
In Tests wurden die Laufzeiten des Gesamtsystems und der einzelnen am System beteiligten Komponenten gemessen und in der Auswertung in Beziehung gesetzt. In einem weiteren Test wurde die Reaktionsf�higkeit des umgesetzten Ansatzes auf Ver�nderungen der Ansicht beim Betrachten untersucht. Es konnte gezeigt werden, dass sich mit dem vorgestellten Out-of-Core-Ansatz, eine schnelle Anpassung der Darstellung an die Ansicht des Betrachters realisieren l�sst.  

\newpage
\section{Ausblick}

Die Verwendung von Sparse Voxel Octrees in Verbindung mit einem leistungsf�higen Out-of-Core-Systems zur Darstellung von hochaufl�sender Geometrie liefert vielversprechende Ergebnisse. Der vor\-ge\-stell\-te Ansatz und dessen Implementation kann an mehreren entscheidenden Stellen optimiert werden. So k�nnten zuk�nftige Arbeiten den Einsatz von OpenGL-Texturen als serverseitigen Incore-Buffer pr�fen, da Zugriffe auf Texturen vom Hardware-Cache der GPU profitieren. Dies sollte sich positiv auf die Bildwiederholrate beim Analyse-Pass und bei der Bildsynthese auswirken.\\
Weitere Entwicklungen k�nnten auf eine Entkoppelung von Out-of-Core-System und der Bilderzeugung zielen, um eine bessere Ausnutzung der Systemresourcen zu gew�hrleisten.
