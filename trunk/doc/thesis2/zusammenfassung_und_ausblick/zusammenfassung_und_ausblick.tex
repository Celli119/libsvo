\chapter{Zusammenfassung und Ausblick}


\section{Zusammenfassung}

%Der Schwerpunkt der vorliegende Arbeit liegt auf...

In der vorliegende Arbeit wurde ein Out-of-Core Ansatz zur Darstellung von gro�en Sparse Voxel Octree Strukturen entwickelt. Dieser arbeitet durch Segmentierung der Octree-Daten in Teile fester Gr��e. Die Anpassung der darzustellenden Untermenge an Octree-Daten wurde durch ein Analysesystem realisiert das auf Abtastung der momentan vorgehaltenen Daten basiert.



Zur Erstellung von Inhalten wurde ein generalisiertes System zur Erstellung von Sparse Voxel Octrees aus unterschiedlichen Ausgangsformaten entworfen. Um die Ergebnisse der Erstellung und des Out-of-Core-Ansatzes zu validieren wurde der Raycasting-Algorithmus in OpenCL implementiert.\\
\\
Es konnte gezeigt werden, dass sich mit dem vorgestellten Out-of-Core-Ansatz eine schnelle Anpassung der Darstellung des Octrees an die Ansicht des Betrachters m�glich ist. 


Die Tests haben gezeigt ... 


\section{Ausblick}
