\section{Kurzfassung}

Raycasting von Sparse Voxel Octrees bietet gegen�ber der Rasterisierung von Dreiecken einige Vorteile. Beispielsweise kann der Detailgrad der dargestellten Geometrie und der zugeh�rigen Attributwerte abh�ngig von Bildaufl�sung und Abstand zum Geometriedetail auf effiziente Weise pro berechneten Bildpunkt gew�hlt werden. Die regul�re Struktur des Octrees macht eine Segmentierung der Daten und eine effiziente Auswahl der  Segmente in Abh�ngigkeit ihres Beitrages zur Qualit�t der Darstellung m�glich. Damit eignet sich diese Datenstruktur besonders f�r hochaufl�sende Geometrie deren Speicher\-gr��e die F�higkeiten aktueller Grafik-Hardware �bersteigt.
\\
\\
In der vorliegende Arbeit wurde ein Out-of-Core-Ansatz zur Darstellung von gro�en Sparse-Voxel-Octree-Strukturen entwickelt. Dieser arbeitet durch Segmentierung der Octree-Daten in Teilen fester Gr��e. Die Anpassung der darzustellenden Untermenge an Octree-Daten wurde durch ein Analysesystem realisiert das auf Abtastung des momentan vorgehaltenen Octrees basiert. Zur Erstellung von Inhalten wurde ein generalisiertes System zur Erstellung von Sparse Voxel Octrees aus unterschiedlichen Aus\-gangs\-formaten entworfen. Um die Ergebnisse der Erstellung und des Out-of-Core-Ansatzes zu validieren wurde der Raycasting-Algorithmus in OpenCL implementiert.\\
In Tests wurden die Laufzeiten des Gesamtsystems und der einzelnen am System beteiligten Komponenten gemessen und in der Auswertung in Beziehung gesetzt. In einem weiteren Test wurde die Reaktionsf�higkeit des umgesetzten Ansatzes auf Ver�nderungen der Ansicht beim Betrachten untersucht.
